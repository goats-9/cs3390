% To familiarize yourself with this template, the body contains
% some examples of its use.  Look them over.  Then you can
% run LaTeX on this file.  After you have LaTeXed this file then
% you can look over the result either by printing it out with
% dvips or using xdvi.
%

\documentclass[twoside]{article}
%\usepackage{soul}
\usepackage{./lecnotes_macros}


\begin{document}
%FILL IN THE RIGHT INFO.
%\lecture{**LECTURE-NUMBER**}{**DATE**}{**LECTURER**}{**SCRIBE**}
\lecture{CS3390}{Foundations of Machine Learning}{6}{8 September 2023}{P. K.
Srijith}{Gautam Singh}
%\footnotetext{These notes are partially based on those of Nigel Mansell.}

%All figures are to be placed in a separate folder named ``images''

% **** YOUR NOTES GO HERE:

\section{Linear Discriminant Analysis}

In the following, \(r_t\) is an indicator variable to select elements belonging
to a certain class.

\begin{enumerate}
    \item Between class scatter is given by
    \begin{align}
        \brak{m_1-m_2}^2 &= \brak{\vec{w}^\top\vec{m_1} -
        \vec{w}^\top\vec{m_2}}^2 \\
                         &= \vec{w}^\top \brak{\vec{m_1-m_2}}
                         \brak{\vec{m_1}-\vec{m_2}}^\top \vec{w} \\
                         &= \vec{w}^\top \vec{S_Bw}
                         \label{eq:bw-scatter}
    \end{align}
    where we define
    \begin{equation}
        \vec{S_B} \triangleq \brak{\vec{m_1} - \vec{m_2}} \brak{\vec{m_1} -
        \vec{m_2}}^\top.
        \label{eq:S-b-def}
    \end{equation}

    \item For any class, within class scatter is given by
    \begin{align}
        s_i^2 &= \sum_t \brak{\vec{w}^\top\vec{x_t} - m_1}^2 r_t \\
              &= \sum_t \vec{w}^\top \brak{\vec{x_t}-\vec{m_1}}
              \brak{\vec{x_t}-\vec{m_1}}^\top \vec{w} r_t \\
              &= \vec{w}^\top\vec{S_iw}
              \label{eq:in-scatter}
    \end{align}
    where we define
    \begin{equation}
        S_i \triangleq \sum_t \brak{\vec{x_t}-\vec{m_1}}
        \brak{\vec{x_t}-\vec{m_1}}^\top r_t.
        \label{eq:S-i-def}
    \end{equation}

    \item For multiple classes, we define
    \begin{equation}
        \vec{S_w} \triangleq \sum_i \vec{S_i}.
        \label{eq:S-w-def}
    \end{equation}
\end{enumerate}

\end{document}
